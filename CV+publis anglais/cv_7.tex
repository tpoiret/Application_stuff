

%%%%%%%%%%%%%%%%%%%%%%%%%%%%%%%%%%%%%%%%%
% "ModernCV" CV and Cover Letter
% LaTeX Template
% Version 1.11 (19/6/14)
%
% This template has been downloaded from:
% http://www.LaTeXTemplates.com
%
% Original author:
% Xavier Danaux (xdanaux@gmail.com)
%
% License:
% CC BY-NC-SA 3.0 (http://creativecommons.org/licenses/by-nc-sa/3.0/)
%
% Important note:
% This template requires the moderncv.cls and .sty files to be in the same 
% directory as this .tex file. These files provide the resume style and themes 
% used for structuring the document.
%
%%%%%%%%%%%%%%%%%%%%%%%%%%%%%%%%%%%%%%%%%

%----------------------------------------------------------------------------------------
%	PACKAGES AND OTHER DOCUMENT CONFIGURATIONS
%----------------------------------------------------------------------------------------

\documentclass[11pt,a4paper,sans]{moderncv} % Font sizes: 10, 11, or 12; paper sizes: a4paper, letterpaper, a5paper, legalpaper, executivepaper or landscape; font families: sans or roman

\moderncvstyle{casual} % CV theme - options include: 'casual' (default), 'classic', 'oldstyle' and 'banking'
\moderncvcolor{blue} % CV color - options include: 'blue' (default), 'orange', 'green', 'red', 'purple', 'grey' and 'black'

\usepackage{lipsum} % Used for inserting dummy 'Lorem ipsum' text into the template

\usepackage[utf8]{inputenc}
\usepackage[T1]{fontenc}
\usepackage[frenchb]{babel}
\usepackage{changepage}

\usepackage[scale=0.75]{geometry} % Reduce document margins
%\setlength{\hintscolumnwidth}{3cm} % Uncomment to change the width of the dates column
%\setlength{\makecvtitlenamewidth}{10cm} % For the 'classic' style, uncomment to adjust the width of the space allocated to your name

%----------------------------------------------------------------------------------------
%	NAME AND CONTACT INFORMATION SECTION
%----------------------------------------------------------------------------------------

\firstname{Thibault} % Your first name
\familyname{Poiret} % Your last name

% All information in this block is optional, comment out any lines you don't need
\title{Curriculum Vitae}
\address{De Genestetstraat 68}{2321 XS Leiden}
\mobile{+336 38 41 08 86}
%\phone{(000) 111 1112}
%\fax{(000) 111 1113}
\email{thibault.poiret5@gmail.com}
%\homepage{staff.org.edu/~jsmith}{staff.org.edu/$\sim$jsmith} % The first argument is the url for the clickable link, the second argument is the url displayed in the template - this allows special characters to be displayed such as the tilde in this example
%\extrainfo{additional information}
\photo[70pt][0.4pt]{pictures/photo2} % The first bracket is the picture height, the second is the thickness of the frame around the picture (0pt for no frame)
%\quote{"A witty and playful quotation" - John Smith}

%----------------------------------------------------------------------------------------

\begin{document}

\makecvtitle % Print the CV title

Born on 04/12/1993.

%----------------------------------------------------------------------------------------
%	EDUCATION SECTION
%----------------------------------------------------------------------------------------

\section{Jobs}

\cventry{2022}{Postdoctoral fellow}{University of Cambridge}{}{}{}
\cventry{2020-2021}{Lecturer}{Universiteit Leiden}{}{}{}


\section{Formation}


\cventry{2016-2020}{PhD in mathematics}{Institut mathématique de Bordeaux, universiteit Leiden}{supervised by Qing Liu and Bas Edixhoven}{}{}

\cventry{2015-2016}{Master's degree in mathematics}{ENS Rennes, université de Rennes 1}{}{}{}

\cventry{2015}{Agrégation of mathematics, computer science option}{}{}{}{}

\cventry{2014-2015}{Preparation for Agrégation}{ENS Rennes, université de Rennes 1}{}{}{}

\cventry{2013-2014}{Bachelor's degree in computer science}{ENS Rennes, université de Rennes 1}{}{}{}


\cventry{2012-2013}{Bachelor's degree in mathematics}{ENS Rennes, université de Rennes 1}{}{}{}

\cventry{2012}{Admission at Ecole Normale Supérieure de Rennes}{}{}{}{}

\cventry{2010-2012}{Preparatory classes}{Lycée Condorcet}{Paris, France}{}{}

\cventry{2010}{Scientific high school diploma}{Lycée Evariste Galois}{Sartrouville, France}{}{}







%----------------------------------------------------------------------------------------
%	WORK EXPERIENCE SECTION
%----------------------------------------------------------------------------------------


\section{Publications}

\subsection{Published}

\cvlistitem{Néron models of Jacobians over bases of arbitrary dimension. epiga:7340 - Épijournal de Géométrie Algébrique, September 21, 2022, Volume 6 - \url{https://epiga.episciences.org/10068}}

\subsection{Submitted}

\cvlistitem{Models of Jacobians of curves, with David Holmes, Samouil Molcho and Giulio Orecchia. In revision after first report from Crelle. \url{https://arxiv.org/pdf/2007.10792.pdf}.}

\subsection{PhD thesis}

\cvlistitem{Néron models in high dimension: Nodal curves, Jacobians and tame base change \url{https://openaccess.leidenuniv.nl/handle/1887/137218}}

\subsection{In preparation}

\cvlistitem{Explicit integral saturation and extended tropicalization, with Dhruv Ranganathan (provisional title)}
\cvlistitem{N\'eron models of curves over locally factorial schemes}
\cvlistitem{Log line bundles and tautological maps of curves}

\subsection{Vague projects}

\cvlistitem{Extending torsors on the moduli space of curves, with Sara Mehidi}
\cvlistitem{The logarithmic structure sheaf, with Patrick Kennedy-Hunt}

\section{Teaching}

\cventry{2016-2017}{Coloration math\'ematique}{Universit\'e de Bordeaux}{}{}{Master class and exercise sessions, first year of Bachelor's}

\cventry{2017}{Algebraic structures}{Universit\'e de Bordeaux}{}{}{Exercise sessions, second year of Bachelor's}

\cventry{2018-2019}{Algebraic geometry 2}{Universiteit Leiden}{}{}{Grading and office hours, second year of Master's}

\cventry{2020}{Analytic number theory}{Universiteit Leiden}{}{}{Exercise sessions and grading, second year of Master's}

\cventry{2022}{Semistable reduction theorems}{University of Cambridge}{}{}{Essay set for the Part III Master students}

\section{Miscellaneous}

\cvitem{}{In 2023 I will be co-organising a lecture group on logarithmic abelian varieties at the university of Cambridge. Below is a non-exhaustive list of topics I have talked about at conferences or seminars.}

\cventry{}{Introduction to elliptic curves}{}{}{}{}

\cventry{}{Introduction to toric varieties}{}{}{}{}

\cventry{}{Deformations of Galois representations}{}{}{}{}

\cventry{}{Néron models of curves over Dedekind schemes}{}{}{}{}

\cventry{}{The tropical Picard group and the tropical Jacobian}{}{}{}{}

\cventry{}{N\'eron models of nodal curves and their Jacobians}{}{}{}{}

\cventry{}{The log Picard group as a N\'eron model}{}{}{}{}

\section{Practical skills}

\subsection{Tongues spoken}

\cvitem{French}{Mother tongue}
\cvitem{English}{Fluent}
\cvitem{Dutch}{Notions}

\subsection{Software tools mastered}

\cvitem{\LaTeX}{Daily usage}
\cvitem{Python}{Intermediate}
\cvitem{Caml}{Intermediate}
\cvitem{Maple}{Notions}
\cvitem{Scilab}{Notions}
\cvitem{Sage}{Notions}
\cvitem{Ruby}{Notions}



\end{document}